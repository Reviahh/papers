\section{Ethical Considerations, Limitations, and Conclusion}
\label{sec:con}
\paragraph{Ethical considerations and limitations}
CausalVerse is a simulated dataset generated using tools like Blender and Unreal Engine 4, containing no real-world or personal data, thus avoiding privacy or consent concerns. While care was taken to avoid biased or anthropomorphized representations, users should note that models trained on synthetic data may not generalize directly to real-world settings. Additionally, CausalVerse’s causal structures are handcrafted and may not capture the full complexity or ambiguity of real-world systems. Despite our efforts to make the simulations as realistic as possible, there is still a gap between simulated data and realistic scenarios. The dataset also lacks natural noise factors such as sensor variability, and its visual complexity remains bounded by simulation capabilities. While full realism cannot yet be achieved, we are committed to moving in that direction by providing the research community with datasets that are as realistic, diverse, and large-scale as possible to support continued progress in CRL. While current rendering technologies still have limitations, we are fortunate to be at a time of rapid advancement in simulation, physically based rendering, and AI-generated content. These developments offer new opportunities to push the boundaries of synthetic realism. 

\paragraph{Conclusion} We present CausalVerse, a large-scale, high-fidelity benchmark for causal representation learning that aims to reconcile both realism with controllability and ground-truth access. By spanning a wide spectrum of domains, from static image generation to multi-agent traffic interactions, CausalVerse enables researchers to evaluate CRL methods across a broad range of diverse and challenging conditions. With fine-grained access to the underlying causal structures and simulation parameters, it further supports both the principled evaluation under idealized assumptions and practical stress testing in complex, realistic settings. Through empirical evaluation of representative approaches, we provide insights into the current state of CRL and its sensitivity to both satisfied and violated assumptions. We hope that CausalVerse serves as a stepping stone toward more reliable, interpretable, and generalizable causal learning systems, and as a foundation for future work that unites causal theory with complex visual environments.