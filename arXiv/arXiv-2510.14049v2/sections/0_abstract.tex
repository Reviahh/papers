

Causal Representation Learning (CRL) aims to uncover the data-generating process and identify the underlying causal variables and relations, whose evaluation remains inherently challenging due to the requirement of known ground-truth causal variables and causal structure. Existing evaluations often rely on either simplistic synthetic datasets or downstream performance on real-world tasks, generally suffering a dilemma between realism and evaluative precision. In this paper, we introduce a new benchmark for CRL using high-fidelity simulated visual data that retains both realistic visual complexity and, more importantly, access to ground-truth causal generating processes. The dataset comprises around 200 thousand images and 3 million video frames across 24 sub-scenes in four domains: static image generation, dynamic physical simulations, robotic manipulations, and traffic situation analysis. These scenarios range from static to dynamic settings, simple to complex structures, and single to multi-agent interactions, offering a comprehensive testbed that hopefully bridges the gap between rigorous evaluation and real-world applicability. In addition, we provide flexible access to the underlying causal structures, allowing users to modify or configure them to align with the required assumptions in CRL, such as available domain labels, temporal dependencies, or intervention histories. Leveraging this benchmark, we evaluated representative CRL methods across diverse paradigms and offered empirical insights to assist practitioners and newcomers in choosing or extending appropriate CRL frameworks to properly address specific types of real problems that can benefit from the CRL perspective. Welcome to visit our: \noindent\textbf{Project page:} \href{https://causal-verse.github.io/}{causal-verse.github.io},
\noindent\textbf{Dataset:} \href{https://huggingface.co/CausalVerse}{huggingface.co/CausalVerse}.

% \noindent\textbf{Code:} \href{https://causal-verse.github.io/index.html}{causal-verse.github.io/index.html}

%or finding suitable abstractions of micro variables. In this paper, we focus on the former type of CRL, in which the valuation of CRL methods 