% A4纸,小四号
\documentclass[a4paper,AutoFakeBold={2},12pt]{ctexrep}

\usepackage{style}
\usepackage{listings}
\usepackage{graphicx}
\usepackage{multirow}
\usepackage{subcaption}
\usepackage{url}
\usepackage{amsmath}
\usepackage{amssymb}
\usepackage{amsfonts}
\usepackage{hologo}
\usepackage{geometry}
\usepackage{fancyhdr}
\usepackage{amsmath}  % 加载amsmath包以支持数学公式
\usepackage{newtxmath}  % 使用 Times New Roman 字体设置数学公式
\usepackage{times}  % 设置正文字体为 Times New Roman
\usepackage{ctex}
\usepackage{fontspec}  % 用于设置字体
\usepackage{titlesec}  % 用于自定义章节标题格式
\usepackage{setspace}  % 用于设置行距
\usepackage{graphicx}  % 用于支持中文字体的加载
\usepackage{xeCJK}     % 处理中文字体
\usepackage{caption}
\usepackage{setspace}
\usepackage{titletoc}
\usepackage{listings}
\usepackage{color}

% 设置页眉
\pagestyle{fancy}

% 设置论文的页面格式
\geometry{
	top=3.3cm,             
	bottom=2.7cm,          
	left=2.5cm,           
	right=2.5cm,            
	bindingoffset=0.5cm,    
}

% 设置英文字体为Times New Roman
\setmainfont{Times New Roman}
\newfontface\timesnewroman{Times New Roman}

% 设置图标题的格式
\captionsetup[figure]{ 
	font={small},         
	labelfont={bf},          
	singlelinecheck=false,  
	justification=centering, 
	skip=0.5\baselineskip, 
}

% 设置表格标题的格式
\captionsetup[table]{
	font={small},  
	labelfont={bf},          
	skip=1ex, 
	justification=centering,  
    aboveskip=0.5\baselineskip,  
    belowskip=0.6\baselineskip,  
}

% 设置表格和图的编号格式 
\renewcommand{\thefigure}{\thechapter.\arabic{figure}}
\renewcommand{\thetable}{\thechapter.\arabic{table}} 

% 设置代码块的格式
\lstset{
	basicstyle=\ttfamily\fontspec{Times New Roman}, 
	keywordstyle=\normalfont,  
	commentstyle=\normalfont, 
	stringstyle=\normalfont, 
	showstringspaces=false, 
	breaklines=true,  
	frame=single, 
	framerule=0.12mm,  
	framesep=1mm, 
}

% 设置论文正文的标题格式
\titleformat{\chapter}  
{\centering\normalfont\bfseries\heiti\fontsize{18pt}{10pt}\selectfont} 
{第\thechapter 章}{1em}{} 
\titlespacing*{\chapter}{0pt}{-1em}{1em}

% 设置论文正文的section标题格式
\titleformat{\section}[hang]     
{\normalfont\bfseries\songti\fontsize{15}{17}\selectfont}
{\thesection}{1em}{}

% 设置论文正文的subsection标题格式
\titleformat{\subsection}[hang]   
{\normalfont\bfseries\songti\fontsize{12}{16}\selectfont}
{\thesubsection}{1em}{}

    
% 设置目录的格式
\renewcommand{\contentsname}{\centering \heiti \zihao{-2} \bfseries 目~~ 录}

\titlecontents{chapter}[0em]{\bfseries\songti\zihao{4}}{\thecontentslabel}{\hspace*{0em}}{\titlerule*[0.6pc]{$.$}\contentspage}
\titlecontents{section}[3em]{\songti\zihao{4}\vspace{6pt}}{\contentslabel{1.6em}}{\hspace*{-4em}}{~\titlerule*[0.6pc]{$.$}~\contentspage}
\titlecontents{subsection}[5em]{\songti\zihao{4}\vspace{5pt}}{\contentslabel{2.3em}}{\hspace*{-4em}}{~\titlerule*[0.6pc]{$.$}~\contentspage}


\begin{document}

\newpage
% 切换到正文的页面样式
\pagestyle{mainmatter}
% 让页码从1开始
\setcounter{page}{1}
% 摘要与目录使用罗马数字页码
\pagenumbering{roman}
\renewcommand{\thepage}{\Roman{page}}

\newpage
\linespread{1.5}

\begin{center}
	% 设置小二号字体和黑体
	{\heiti \zihao{-2} \textbf{中文题目}}
\end{center}

\begin{center}
	% 设置小四号宋体,文本加粗,且“摘要”两个字中间空两个空格
	 {\songti \zihao{4} \textbf{摘~  要}}
\end{center}

{
\zihao{-4}

随着科技的不断发展,人工智能在各个领域的应用越来越广泛。尤其是在医疗、金融和教育等行业,人工智能技术已经逐渐渗透并发挥着重要作用。许多企业和科研机构也开始加大对人工智能技术的投入,推动了这一领域的迅速发展。未来,人工智能有望带来更多创新,改变我们的工作和生活方式。

在人工智能的众多应用中,机器学习和深度学习无疑是最为重要的技术之一。通过大数据和算法的不断优化,机器学习能够使系统在没有明确编程的情况下,从数据中学习规律并做出预测。深度学习则进一步提升了模型的表现,尤其是在图像识别、语音处理和自然语言处理等方面取得了显著进展。

然而,随着人工智能技术的快速发展,也带来了许多挑战和问题。例如,如何避免算法的偏见,如何保障数据隐私和安全,如何平衡技术发展与伦理问题等,都是当前亟需解决的重要议题。未来,随着人工智能在各行各业的深入应用,我们需要更加关注其社会影响,确保技术为人类带来积极的改变。

{\songti \zihao{-4} \textbf{关键词:}}XXXXX;XXXXXX;XXXXXX
}

(\textcolor{red}{关键词3-5个, 用分号“;”隔开, 不能全部采用英文缩写})

\newpage

\begin{center}
    \zihao{-2}
    \textbf{English Title}
\end{center}

\begin{center}
    \zihao{4}
    \textbf{Abstract}
\end{center}

{
\zihao{-4}

In recent years, the field of artificial intelligence has made significant advancements, particularly in the areas of machine learning and deep learning. These technologies have revolutionized a wide range of industries, from healthcare to finance, enabling more efficient processes and better decision-making. As a result, there is a growing demand for skilled professionals who can harness the power of AI to solve real-world problems.

One of the most exciting developments in AI is the progress made in natural language processing (NLP). With the ability to understand, interpret, and generate human language, NLP has paved the way for chatbots, virtual assistants, and machine translation tools. These advancements have made it possible to interact with computers in more natural and intuitive ways, enhancing user experience and making technology more accessible to a wider audience.

Despite the tremendous potential of AI, there are still significant challenges that need to be addressed. Ethical considerations, such as bias in algorithms and the impact on employment, remain critical issues. Moreover, as AI systems become more complex, ensuring transparency and accountability in their decision-making processes is essential. Therefore, ongoing research and collaboration between experts from various fields are crucial to ensuring the responsible development and deployment of AI technologies.

\textbf{Keywords:} LaTeX; thesis; tutorial
}

(\textcolor{red}{英文标题和摘要由对应的中文标题和摘要翻译而来,关键词3-5个,用分号“;”分隔})




\vspace*{-1.5cm}  % 调整“目录”与页眉之间的间距
\setstretch{1.5}
\tableofcontents

\setstretch{1.5}
\chapter*{前~~ 言}  % 章节不编号
\addcontentsline{toc}{chapter}{前~~ 言}  % 手动将“前言”添加到目录中


% 让页码从1开始

% 正文使用阿拉伯数字页码
\pagenumbering{arabic}


\hspace*{2em} \textcolor{red}{前言内容部分保持在1-2个page。简单介绍论文研究背景、研究意义、引出本论文主要工作。}

\textcolor{red}{目录层面,采用自动生成方式生成目录,目录仅包含后续的正文部分的三级标题。} 

\chapter{绪~ 论}

\hspace*{2em}本章首先陈述了夜景图像增强的研究背景和意义,其次简单介绍了零参考深度曲线估计的原理和其优势,并概述了本文完成的主要工作和贡献。最后,本章还展示了本文的整体组织结构。

\section{研究背景及意义}

随着计算机视觉和人工智能研究的不断发展,该领域取得了前所未有的进步。从普通的人工智能任务,如目标识别$^{\textnormal{[1]}}$,图像分类$^{\textnormal{[2]}}$,到更为复杂的AI任务,例如围棋学习$^{\textnormal{[3]}}$,回答阅读理解问题,回答图像或视频的问题等。\textcolor{red}{(参考文献的标注要与参考文献.tex要一致,同时注意参考文献的标记要用上标。)}


\section{研究现状}

针对论文研究的方法或者开发系统现有的成果进行综述分析,并最后给出目前存在的问题或不足。

\section{论文主要研究工作}

本文以基于学习的夜景图像增强为问题导向,主要以零参考深度曲线估计方法为研究对象,分析了现有的几种夜景图像增强算法,对比检测了它们的优缺点。在深入分析了零参考深度曲线估计方法的源代码的基础上,对其进行了消融实验以测试各损失函数的作用,实验测试了它在不同类型数据集上的增强效果,用不同种类的训练集和测试集来测试其拟合情况。对于该方法欠缺考虑的噪声问题,本文优化了其源码,在损失函数中加入了关于图像噪声的损失,并实验得出了这一损失在总损失中比较合适的权重。最后,以不同数据集训练,得出了一种令其表现出色的训练数据集选择方法。本文的主要工作及贡献如下:

(1)分析了夜景图像增强相比于一般图像增强的难点。

(2)分析了传统的夜景图像增强方法和基于深度学习的图像增强方法,对比它们的优势与不足,总结了基于学习的夜景图像增强算法的优点。

(3)分析了零参考深度曲线方法的原理和优缺点,对其进行复现并部署在云端GPU平台,以消融实验测试其损失函数作用,测试不同训练集对其增强结果的影响,评估其是否出现过拟合现象。

(4)搜集数据集,编写程序对数据集进行分类,将不可用或部分可用图像数据集转换为可用数据集,测试不同数据集的训练效果及最终方法在不同测试集上的效果。

(5)完成对Zero-DCE的优化改善,让其增强结果的噪声大幅下降。主要通过补充其损失函数完成改进。完成了补充损失后的代码,并以实验得出了该损失在总损失中的合适权重。

\section{论文结构安排}

本文共分为六章,各章内容安排如下:

第一章绪论介绍了本文所述课题的研究背景和意义,简单地介绍了卷积神经网络以及本文所研究算法的核心深度曲线网络,本文完成的主要工作和贡献,最后介绍本文的组织结构。

第二章相关基础知识概述,阐述了数字图像的一些基本属性,图像增强的原理以及夜景图像增强的难点,然后介绍了用于图像增强的传统方法和它们的优缺点。

第三章首先介绍了深度学习的一些常用方法和相关概念,然后详细分析了基于GAN的方法,详细剖析了用于低光图像增强的零参考深度曲线估计的原理和它的实现方式,提出了对它的改进和改进的实现。

第四章描述了改进后的用于低光图像增强的零参考深度曲线估计在云端GPU的部署实现,并以实验测试了其在不同数据集上训练后的效果,改进后的性能以及它的各部分损失函数的作用,最后对课题的实现进行了可视化展示。

第五章总结全文,提出了一些关于该课题的未来工作,可补充内容以及展望。

\chapter{XXXXXXXXXXXXXX}

\hspace*{2em}本章先介绍了数字图像的相关概念,再阐述图像增强的概念和其原理,最后分析了夜景图像增强相对于一般图像增强的难点。

\section{XXXXXXXX}

数字图像由基本的像素构成。像素的由来是将模拟图像数字化时对连续的实体离散化,如图2.1所示。像素拥有其行列的位置坐标信息,还有表示它亮度的灰度值,这个灰度值通常是对应于一个通道的(三通道图像在一个位置就有三个灰度值),且一般是整数,这些值常被压缩后存储以节省空间。
\begin{figure}[h]
	\centering 
	\includegraphics[width=0.8\textwidth]{figures/fig1}
	\caption{\textbf{\songti 二维标量场的可视化}}
\end{figure}

\textcolor{red}{图编号以“章编号+序号”方式进行,如第二章第一个图,编号为:“图2.1”。图需要在正文中引用,先文后图,即正文中先出现“如图2.1所示”,再文字下方在给出图。如正文示例中标黄色的地方。图不能分页排版,应和文中首次引用位置在同一页。如果图和首次文字引用位置不在同一页,则应将文字引用改为“如下页图2.1所示。”文中第二次引用位置可描述为“如第X页图2.1所示”。图中的文字应比正文小。四周不可留较多空白,需要适当剪切。}

\subsection{XXXXXXXXXX}

视觉对话的常用数据集为VisDial v1.0。它基于MSCOCO数据集的标题和图像进行收集,其中,图片对应的对话由两人通过提问和回答的方式进行收集,对于数据集中的每张图片而言,提问者只能看到标题和对话历史,而回答者可以看到标题、历史和图像。每张图片的对话由10轮问答组成。任何一个当前问题的答案都不包括在对话历史中。VisDial v1.0分为训练集、验证集和测试集3个子集。其中,训练集包括123287个对话,验证集包括2064个对话,测试集包含8000个对话。

此外,VisDial的旧版本v0.9也常用于视觉对话,由于它包含的对话和图片较少,数据集的质量也不如v1.0版本,很多模型已经弃用了它,但是它还是可以用来评估模型的质量,并且它数据集较小,模型训练起来更加方便。它共包含1.4M的问答对。

\subsection{XXXXXXXX}

视觉对话的评估方法借鉴了检索的评估方法。每个问题的候选答案有100个,模型需要返回这100个候选答案的排序。模型的评估指标有两类,对于候选答案中只有一个正确答案的答案注释,评估指标包括标准答案在前k个答案中响应的比例(Recall@K),标准答案的平均排序(Mean)(越低越好),平均倒数排序(MRR)。其中MRR是对所有正确答案在模型结果中的排序去倒数后在求取平均值,结果越高越好,该指标侧重于人类的真实答案,缺点是会忽略很多其他可能正确的答案。第二类评估指标基于密集的答案注释,候选答案中有数个正确答案,正确程度通过值域为(0,1)的相关性比率来表示,评估指标为归一化折现累计增益(NDCG),由于该指标依赖第三人标记所有的正确答案,所以对于不确定性的问题,该指标效果更佳。

\section{XXXXXXXXXXXXXX}

\subsection{XXXXXXXXXXX}

\subsection{XXXXXXXXXXXXXX}

上述是一些最常用的库,它们的有些功能和模块都类似于表2.1中Scikit-image的,所以这里不做过多阐释。但实际应用中它们各有所长,可以依据环境和测试结果进行选择。同样的图像增强算法往往可以用不同的库来实现,它们的运行效率和效果一般不同,具体选择要以实验结果为准。

\begin{table}[h]
	\centering
	\caption{\textbf{Scikit-image \songti 常用子模块及其功能}}
	\small
	% 设置表格行间距为1.5倍
	\renewcommand{\arraystretch}{1.5}
	
	\begin{tabular}{|c|c|}
		\hline
		子模块 & 功能 \\
		\hline
		`skimage.feature` & 提供计算图像特征(如纹理、边缘、角点检测等)的方法。 \\
		\hline
		`skimage.filters` & 提供多种图像滤波操作,如平滑、边缘检测等。 \\
		\hline
		`skimage.transform` & 提供图像的几何变换方法,如旋转、缩放、仿射变换等。 \\
		\hline
		`skimage.color` & 用于颜色空间转换,如从RGB转换到灰度图像。 \\
		\hline
		`skimage.measure` & 提供图像区域的测量与分析功能,如连通区域分析、轮廓提取等。 \\
		\hline
		`skimage.io` & 用于图像的读取、保存和显示功能。 \\
		\hline
		`skimage.restoration` & 提供图像去噪和恢复的方法,如去模糊、去噪等。 \\
		\hline
	\end{tabular}
\end{table}

\textcolor{red}{表编号以“章编号+序号”方式进行,如第二章第一个表,编号为:“表2.1”。表需要在正文中引用,先文后表,即正文中先出现“表2.1”,再文字下方在给出表。如正文示例中标黄色的地方。表一般不分页排版,应和文中首次引用位置在同一页。如果表和首次文字引用位置不在同一页,则应将文字引用改为“如下页表2.1所示。”文中第二次引用位置可描述为“如第X页表2.1所示”。表中的文字比正文小一号,或根据需要采用更小的字体,但确保能看得清楚。}


\section{XXXXXXXXXXXXXXXXX}

\section{XXXXXXXXXXXXXXXXXXXXX}

\subsection{XXXXXXXXXXXXXXXX}

文献[8]提出了一种XXXXX方法。

直方图均衡化$^{\textnormal{[8]}}$是一种常用的灰度值变换方法。对于一个数字图像,它的直方图可表示为离散函数,如公式(2.1)所示:
\begin{equation}
	h(k) = n_k
\end{equation}

其中,$k$是灰度值,$n_k$是该灰度值的像素个数。

\textcolor{red}{参考文献的引用,如标注位置所示:A. 如果是正文中的描述,则采用正文方式,如文中“文献[8]提出了一种……”; B. 如果不是正文,采用上标方式标注。}

\textcolor{red}{公式需要在正文中引用,先文后式,即正文中先出现“如公式2.1”,在文字下方在给出公式。公式中的符号书写需要和正文对应符号保持一致。可以使用网上在线的Latex公式编辑器或者转换器转换。}

\subsection{xxxxxxxxxxx}

\section{本章小结}

本章简单介绍了图像处理领域的相关基础知识,从数字图像的基本属性、常用格式,到增强的原理,总结了夜景图像增强的难点,最后介绍了增强的传统方法。

\chapter{xxxxxxxxxxxxxxx}

\hspace*{2em}本章首先介绍深度学习相关知识和其在Python中常用库,库中的相关知识,再介绍几种本文所研究过的基于深度学习夜景图像算法,对比其增强效果。

\section{XXXXXXXXx}

\subsection{XXxxx}

\subsection{XXXXXX}

\section{XXXXXXx}

\section{XXXXXXXX}

\subsection{XXXXXX}

这样的二次曲线,可以表示为,如公式(3.1)所示:
\begin{equation}
	E = mc^2
\end{equation}
\hspace*{2em}其中,$E$ 为能量,$m$ 为质量,$c$ 为光速。公式(3.1)中定义的LE-curve可以迭代应用,以实现更通用的调整,以应对具有挑战性的低光条件。

\subsection{XXXXXXXXXXXXXXX}

\subsection{xxxxxxxxxxxxxxxxxx}

本章首先介绍了深度学习的相关概念和方法,然后分析了基于GAN的增强方法,又深入剖析了Zero-DCE的原理$^{\textnormal{[5]}}$,在发现其噪声处理方面的不足后提出了改进方法,下一章将具体阐述对改进后的该方法进行的相关实验。于是我基于该指标编写了损失函数的相关类,其核心代码如下:
\begin{lstlisting}[language=Python]
class L_psnr(nn.Module):
   def __init__(self):
     super(L_psnr, self).__init__()
  
   def forward(self, org, enhenced):
     mse = torch.mean((org / 255. - enhenced / 255.) ** 2)
     if mse < 1.0e-10:
        return 100
     PIXEL_MAX = 1       
     return 20 * math.log10(PIXEL_MAX / math.sqrt(mse))
\end{lstlisting}

\textcolor{red}{核心代码的书写示例,可以根据自己的习惯调整}

\textcolor{red}{参考链接:https://blog.csdn.net/horsee/article/details/128623180}

\section{本章小结}

本章首先介绍了深度学习的相关概念和方法,然后分析了基于GAN的增强方法,又深入剖析了Zero-DCE的原理$^{\textnormal{[5]}}$,在发现其噪声处理方面的不足后提出了改进方法,下一章将具体阐述对改进后的该方法进行的相关实验。

\chapter{XXXXX实现与实验结果分析}

\hspace*{2em}本章在深入分析了基于注意力的双视图网络模型的原理后,对该方法进行了复现,并扩展了其数据集,在VisDial v0.9上也做了训练和评估,然后将上节提到的优化器和激活函数的改进写入其中,并以消融实验验证了它们的效果,之后测试了位置编码和损失函数的实际效果,最后进行了部分结果的可视化展示。

\section{XXXXX实现}

\subsection{平台与环境配置}

由于该模型网络结构复杂,所采用的VisDial v1.0数据集规模庞大,大小约100G,所以起初在自己的1050ti上训练极为缓慢,训练了38.5小时后才训练了6个轮次,还发生显存溢出的错误,所以之后我在AutoDL的云端GPU上运行,用了两块2080tiGPU,完成一次训练大概8小时,尽管还是缓慢,但勉强可以接受。实验环境如下:GPU为RTX2080ti(11GB)*2,CPU为8核Intel(R) Xeon(R) Silver 4110 CPU @ 2.10GHz,实际Conda环境下CUDA版本为10.1,Python版本3.7。
\subsection{问题和解决方法}

\subsection{实现过程}

\section{实验数据集介绍}

\section{实验结果与分析}

\section{结果可视化展示}

\section{本章小结}
本文在深入分析了基于注意力的双视图网络模型后,完成了对该模型的改进,本章通过实验测试了它的可行性,效果和缺陷。实验表明,优化器和激活函数的改进取得了不错的效果,而模型输入编码和损失函数的改进存在局限性,体现在训练速率降低或在训练多轮次后改进效果变差或者没有明显改进上,但是在进行短期且灵活的训练和评估时效果可以接受。

\chapter{总结与展望}

\section{本文工作总结}

本文首先介绍了视觉对话任务并分析其难点,然后介绍了主流的视觉对话模型类别并挑选代表性模型分析其特点。然后,本文重点分析了基于注意力的双视图网络模型的框架和优点,深入剖析了其各个模块的构建细节和作用,之后完成了对它的测试,优化和模型改进,取得较为满意的效果,最终完成了基于注意力的视觉对话模型架构。
本文的主要工作及贡献如下:

(1)分析了主流的各种视觉对话模型,对每一类挑选代表性的模型对比总结它们的特点,并选定了本文研究的基模型。

(2)分析了基于注意力的双视图网络模型的原理,框架模块及优缺点,基于该模型提供的模板代码对其进行复现并部署在云端GPU平台。

\section{后期工作展望}

在进行了大量的实验后,发现位置编码的添加所取得的评估结果不能令人满意,尤其在进行大量轮次的训练后评估指标不如原模型,因此位置编码的应用细节仍然需要大量工作来测试和优化。

\textcolor{red}{"总结与展望"这一章篇幅至少1个完整的页面,建议2页。}

\textcolor{red}{参考文献以正文引用的先后顺序进行编号,这里给出了中英文期刊、会议文章以及学位论文的文献格式样例,正文中的所有参考文献请参考该样例撰写: A. 建议参考文献在25-30条为宜; B. 参考文献列表顺序请按照在正文引用标注的先后顺序进行排列,所有文献必须在正文中引用;C. 建议有适当的中文参考文献;D. 参考文献最好是近5年内的,经典文献除外;E. 参考文献的所有作者均需要列出,不能用et al代替;F. 参考文献条目中的标点符号一律用英文标点,标点后面空一格。}


\addcontentsline{toc}{chapter}{参考文献}

\begin{thebibliography}{99}

    \bibitem{1} 陈佛计, 朱枫, 吴清潇, 郝颖明, 王恩德, 崔芸阁. 生成对抗网络及其在图像生成中的应用研究综述 [J]. 计算机学报, 2021, 44(02): 347-369.
                
    \bibitem{2} Xijuan Song, Jijiang Huang, Jianzhong Cao, Dawei Song. Multi-scale joint network based on Retinex theory for low-light enhancement [J]. Signal, Image and Video Processing, 2021, 2(02): 1-8.
      
    \bibitem{3} Ruixing Wang, Qing Zhang, Chi-Wing Fu, Xiaoyong Shen,Wei-Shi Zheng, and Jiaya Jia. Underexposed photo enhancement using deep illumination estimation [C]. In CVPR, 2019.

    \bibitem{4} 包俊, 董亚超, 刘宏哲. 卷积神经网络的发展综述[C]. 中国计算机用户协会网络应用分会2020年第二十四届网络新技术与应用年会论文集, 2020, 11: 16-21.
    
    \bibitem{5} 赵晋东. 复杂光照条件下的人脸识别方法研究[D]. 中国民航大学, 2020.
    
    \bibitem{6} Jianping Gou, Baosheng Yu, Stephen J. Maybank, et al. Knowledge Distillation: A Survey [J]. International Journal of Computer Vision, 2021, 129(6): 1789–1819.

\end{thebibliography}



\chapter*{致~~~~ 谢}
\addcontentsline{toc}{chapter}{致~~~~~~~ 谢}

\hspace*{2em}致谢部分是对学生在完成毕业设计(论文)过程中给予帮助和支持的人表示感谢的环节。以下是一些常见的需要感谢的对象:

1、指导教师:感谢指导教师在毕设过程中的指导和帮助,包括选题、研究方法、论文写作等方面的建议和指导。

2、同学和室友:感谢同学们在学习、讨论和生活上的支持和帮助,特别是那些在毕设过程中一起讨论问题、分享资料的同学。

3、家人:感谢家人在精神和物质上的支持,以及在毕设期间的理解和鼓励。

4、学院和系部:感谢所在学院或系部提供的资源和支持,包括实验室设备、资料室资源等。

5、参与调查或实验的志愿者:如果毕设涉及到调查问卷或实验,需要感谢那些愿意参与并提供数据的志愿者。

6、其他教师:如果其他教师在毕设过程中提供了帮助,比如提供额外的指导、建议或资源,也应该表示感谢。

7、资助机构:如果毕设有外部资助,需要感谢提供资金支持的机构或个人。

8、任何其他提供帮助的人:比如提供专业建议的行业专家、提供技术支持的IT人员等。
致谢部分应该简洁、真诚,表达对帮助和支持的感激之情。

\appendix

\chapter*{附~~~~ 录}
\setcounter{chapter}{1}


\hspace*{2em} 这部分可以包括论文的符号说明、程序源代码、原始材料、调查报告等。如果没有,这部分可以自行删除。 


\end{document}
