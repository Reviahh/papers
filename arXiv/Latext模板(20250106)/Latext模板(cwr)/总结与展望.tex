\section{本文工作总结}

本文首先介绍了视觉对话任务并分析其难点,然后介绍了主流的视觉对话模型类别并挑选代表性模型分析其特点。然后,本文重点分析了基于注意力的双视图网络模型的框架和优点,深入剖析了其各个模块的构建细节和作用,之后完成了对它的测试,优化和模型改进,取得较为满意的效果,最终完成了基于注意力的视觉对话模型架构。
本文的主要工作及贡献如下:

(1)分析了主流的各种视觉对话模型,对每一类挑选代表性的模型对比总结它们的特点,并选定了本文研究的基模型。

(2)分析了基于注意力的双视图网络模型的原理,框架模块及优缺点,基于该模型提供的模板代码对其进行复现并部署在云端GPU平台。

\section{后期工作展望}

在进行了大量的实验后,发现位置编码的添加所取得的评估结果不能令人满意,尤其在进行大量轮次的训练后评估指标不如原模型,因此位置编码的应用细节仍然需要大量工作来测试和优化。

\textcolor{red}{"总结与展望"这一章篇幅至少1个完整的页面,建议2页。}

\textcolor{red}{参考文献以正文引用的先后顺序进行编号,这里给出了中英文期刊、会议文章以及学位论文的文献格式样例,正文中的所有参考文献请参考该样例撰写: A. 建议参考文献在25-30条为宜; B. 参考文献列表顺序请按照在正文引用标注的先后顺序进行排列,所有文献必须在正文中引用;C. 建议有适当的中文参考文献;D. 参考文献最好是近5年内的,经典文献除外;E. 参考文献的所有作者均需要列出,不能用et al代替;F. 参考文献条目中的标点符号一律用英文标点,标点后面空一格。}
