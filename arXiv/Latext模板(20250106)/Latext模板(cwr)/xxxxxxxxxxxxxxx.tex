\hspace*{2em}本章首先介绍深度学习相关知识和其在Python中常用库,库中的相关知识,再介绍几种本文所研究过的基于深度学习夜景图像算法,对比其增强效果。

\section{XXXXXXXXx}

\subsection{XXxxx}

\subsection{XXXXXX}

\section{XXXXXXx}

\section{XXXXXXXX}

\subsection{XXXXXX}

这样的二次曲线,可以表示为,如公式(3.1)所示:
\begin{equation}
	E = mc^2
\end{equation}
\hspace*{2em}其中,$E$ 为能量,$m$ 为质量,$c$ 为光速。公式(3.1)中定义的LE-curve可以迭代应用,以实现更通用的调整,以应对具有挑战性的低光条件。

\subsection{XXXXXXXXXXXXXXX}

\subsection{xxxxxxxxxxxxxxxxxx}

本章首先介绍了深度学习的相关概念和方法,然后分析了基于GAN的增强方法,又深入剖析了Zero-DCE的原理$^{\textnormal{[5]}}$,在发现其噪声处理方面的不足后提出了改进方法,下一章将具体阐述对改进后的该方法进行的相关实验。于是我基于该指标编写了损失函数的相关类,其核心代码如下:
\begin{lstlisting}[language=Python]
class L_psnr(nn.Module):
   def __init__(self):
     super(L_psnr, self).__init__()
  
   def forward(self, org, enhenced):
     mse = torch.mean((org / 255. - enhenced / 255.) ** 2)
     if mse < 1.0e-10:
        return 100
     PIXEL_MAX = 1       
     return 20 * math.log10(PIXEL_MAX / math.sqrt(mse))
\end{lstlisting}

\textcolor{red}{核心代码的书写示例,可以根据自己的习惯调整}

\textcolor{red}{参考链接:https://blog.csdn.net/horsee/article/details/128623180}

\section{本章小结}

本章首先介绍了深度学习的相关概念和方法,然后分析了基于GAN的增强方法,又深入剖析了Zero-DCE的原理$^{\textnormal{[5]}}$,在发现其噪声处理方面的不足后提出了改进方法,下一章将具体阐述对改进后的该方法进行的相关实验。