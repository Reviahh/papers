\hspace*{2em}本章先介绍了数字图像的相关概念,再阐述图像增强的概念和其原理,最后分析了夜景图像增强相对于一般图像增强的难点。

\section{XXXXXXXX}

数字图像由基本的像素构成。像素的由来是将模拟图像数字化时对连续的实体离散化,如图2.1所示。像素拥有其行列的位置坐标信息,还有表示它亮度的灰度值,这个灰度值通常是对应于一个通道的(三通道图像在一个位置就有三个灰度值),且一般是整数,这些值常被压缩后存储以节省空间。
\begin{figure}[h]
	\centering 
	\includegraphics[width=0.8\textwidth]{figures/fig1}
	\caption{\textbf{\songti 二维标量场的可视化}}
\end{figure}

\textcolor{red}{图编号以“章编号+序号”方式进行,如第二章第一个图,编号为:“图2.1”。图需要在正文中引用,先文后图,即正文中先出现“如图2.1所示”,再文字下方在给出图。如正文示例中标黄色的地方。图不能分页排版,应和文中首次引用位置在同一页。如果图和首次文字引用位置不在同一页,则应将文字引用改为“如下页图2.1所示。”文中第二次引用位置可描述为“如第X页图2.1所示”。图中的文字应比正文小。四周不可留较多空白,需要适当剪切。}

\subsection{XXXXXXXXXX}

视觉对话的常用数据集为VisDial v1.0。它基于MSCOCO数据集的标题和图像进行收集,其中,图片对应的对话由两人通过提问和回答的方式进行收集,对于数据集中的每张图片而言,提问者只能看到标题和对话历史,而回答者可以看到标题、历史和图像。每张图片的对话由10轮问答组成。任何一个当前问题的答案都不包括在对话历史中。VisDial v1.0分为训练集、验证集和测试集3个子集。其中,训练集包括123287个对话,验证集包括2064个对话,测试集包含8000个对话。

此外,VisDial的旧版本v0.9也常用于视觉对话,由于它包含的对话和图片较少,数据集的质量也不如v1.0版本,很多模型已经弃用了它,但是它还是可以用来评估模型的质量,并且它数据集较小,模型训练起来更加方便。它共包含1.4M的问答对。

\subsection{XXXXXXXX}

视觉对话的评估方法借鉴了检索的评估方法。每个问题的候选答案有100个,模型需要返回这100个候选答案的排序。模型的评估指标有两类,对于候选答案中只有一个正确答案的答案注释,评估指标包括标准答案在前k个答案中响应的比例(Recall@K),标准答案的平均排序(Mean)(越低越好),平均倒数排序(MRR)。其中MRR是对所有正确答案在模型结果中的排序去倒数后在求取平均值,结果越高越好,该指标侧重于人类的真实答案,缺点是会忽略很多其他可能正确的答案。第二类评估指标基于密集的答案注释,候选答案中有数个正确答案,正确程度通过值域为(0,1)的相关性比率来表示,评估指标为归一化折现累计增益(NDCG),由于该指标依赖第三人标记所有的正确答案,所以对于不确定性的问题,该指标效果更佳。

\section{XXXXXXXXXXXXXX}

\subsection{XXXXXXXXXXX}

\subsection{XXXXXXXXXXXXXX}

上述是一些最常用的库,它们的有些功能和模块都类似于表2.1中Scikit-image的,所以这里不做过多阐释。但实际应用中它们各有所长,可以依据环境和测试结果进行选择。同样的图像增强算法往往可以用不同的库来实现,它们的运行效率和效果一般不同,具体选择要以实验结果为准。

\begin{table}[h]
	\centering
	\caption{\textbf{Scikit-image \songti 常用子模块及其功能}}
	\small
	% 设置表格行间距为1.5倍
	\renewcommand{\arraystretch}{1.5}
	
	\begin{tabular}{|c|c|}
		\hline
		子模块 & 功能 \\
		\hline
		`skimage.feature` & 提供计算图像特征(如纹理、边缘、角点检测等)的方法。 \\
		\hline
		`skimage.filters` & 提供多种图像滤波操作,如平滑、边缘检测等。 \\
		\hline
		`skimage.transform` & 提供图像的几何变换方法,如旋转、缩放、仿射变换等。 \\
		\hline
		`skimage.color` & 用于颜色空间转换,如从RGB转换到灰度图像。 \\
		\hline
		`skimage.measure` & 提供图像区域的测量与分析功能,如连通区域分析、轮廓提取等。 \\
		\hline
		`skimage.io` & 用于图像的读取、保存和显示功能。 \\
		\hline
		`skimage.restoration` & 提供图像去噪和恢复的方法,如去模糊、去噪等。 \\
		\hline
	\end{tabular}
\end{table}

\textcolor{red}{表编号以“章编号+序号”方式进行,如第二章第一个表,编号为:“表2.1”。表需要在正文中引用,先文后表,即正文中先出现“表2.1”,再文字下方在给出表。如正文示例中标黄色的地方。表一般不分页排版,应和文中首次引用位置在同一页。如果表和首次文字引用位置不在同一页,则应将文字引用改为“如下页表2.1所示。”文中第二次引用位置可描述为“如第X页表2.1所示”。表中的文字比正文小一号,或根据需要采用更小的字体,但确保能看得清楚。}


\section{XXXXXXXXXXXXXXXXX}

\section{XXXXXXXXXXXXXXXXXXXXX}

\subsection{XXXXXXXXXXXXXXXX}

文献[8]提出了一种XXXXX方法。

直方图均衡化$^{\textnormal{[8]}}$是一种常用的灰度值变换方法。对于一个数字图像,它的直方图可表示为离散函数,如公式(2.1)所示:
\begin{equation}
	h(k) = n_k
\end{equation}

其中,$k$是灰度值,$n_k$是该灰度值的像素个数。

\textcolor{red}{参考文献的引用,如标注位置所示:A. 如果是正文中的描述,则采用正文方式,如文中“文献[8]提出了一种……”; B. 如果不是正文,采用上标方式标注。}

\textcolor{red}{公式需要在正文中引用,先文后式,即正文中先出现“如公式2.1”,在文字下方在给出公式。公式中的符号书写需要和正文对应符号保持一致。可以使用网上在线的Latex公式编辑器或者转换器转换。}

\subsection{xxxxxxxxxxx}

\section{本章小结}

本章简单介绍了图像处理领域的相关基础知识,从数字图像的基本属性、常用格式,到增强的原理,总结了夜景图像增强的难点,最后介绍了增强的传统方法。