\hspace*{2em}本章首先陈述了夜景图像增强的研究背景和意义,其次简单介绍了零参考深度曲线估计的原理和其优势,并概述了本文完成的主要工作和贡献。最后,本章还展示了本文的整体组织结构。

\section{研究背景及意义}

随着计算机视觉和人工智能研究的不断发展,该领域取得了前所未有的进步。从普通的人工智能任务,如目标识别$^{\textnormal{[1]}}$,图像分类$^{\textnormal{[2]}}$,到更为复杂的AI任务,例如围棋学习$^{\textnormal{[3]}}$,回答阅读理解问题,回答图像或视频的问题等。\textcolor{red}{(参考文献的标注要与参考文献.tex要一致,同时注意参考文献的标记要用上标。)}


\section{研究现状}

针对论文研究的方法或者开发系统现有的成果进行综述分析,并最后给出目前存在的问题或不足。

\section{论文主要研究工作}

本文以基于学习的夜景图像增强为问题导向,主要以零参考深度曲线估计方法为研究对象,分析了现有的几种夜景图像增强算法,对比检测了它们的优缺点。在深入分析了零参考深度曲线估计方法的源代码的基础上,对其进行了消融实验以测试各损失函数的作用,实验测试了它在不同类型数据集上的增强效果,用不同种类的训练集和测试集来测试其拟合情况。对于该方法欠缺考虑的噪声问题,本文优化了其源码,在损失函数中加入了关于图像噪声的损失,并实验得出了这一损失在总损失中比较合适的权重。最后,以不同数据集训练,得出了一种令其表现出色的训练数据集选择方法。本文的主要工作及贡献如下:

(1)分析了夜景图像增强相比于一般图像增强的难点。

(2)分析了传统的夜景图像增强方法和基于深度学习的图像增强方法,对比它们的优势与不足,总结了基于学习的夜景图像增强算法的优点。

(3)分析了零参考深度曲线方法的原理和优缺点,对其进行复现并部署在云端GPU平台,以消融实验测试其损失函数作用,测试不同训练集对其增强结果的影响,评估其是否出现过拟合现象。

(4)搜集数据集,编写程序对数据集进行分类,将不可用或部分可用图像数据集转换为可用数据集,测试不同数据集的训练效果及最终方法在不同测试集上的效果。

(5)完成对Zero-DCE的优化改善,让其增强结果的噪声大幅下降。主要通过补充其损失函数完成改进。完成了补充损失后的代码,并以实验得出了该损失在总损失中的合适权重。

\section{论文结构安排}

本文共分为六章,各章内容安排如下:

第一章绪论介绍了本文所述课题的研究背景和意义,简单地介绍了卷积神经网络以及本文所研究算法的核心深度曲线网络,本文完成的主要工作和贡献,最后介绍本文的组织结构。

第二章相关基础知识概述,阐述了数字图像的一些基本属性,图像增强的原理以及夜景图像增强的难点,然后介绍了用于图像增强的传统方法和它们的优缺点。

第三章首先介绍了深度学习的一些常用方法和相关概念,然后详细分析了基于GAN的方法,详细剖析了用于低光图像增强的零参考深度曲线估计的原理和它的实现方式,提出了对它的改进和改进的实现。

第四章描述了改进后的用于低光图像增强的零参考深度曲线估计在云端GPU的部署实现,并以实验测试了其在不同数据集上训练后的效果,改进后的性能以及它的各部分损失函数的作用,最后对课题的实现进行了可视化展示。

第五章总结全文,提出了一些关于该课题的未来工作,可补充内容以及展望。