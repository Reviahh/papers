\section{Introduction}

\subsection{Motivation: The Ubiquity of Jitter}

Modern web applications are fundamentally \textit{event-driven}. User interactions—keystrokes, mouse movements, scroll events, window resizing—generate streams of events that trigger computational responses. However, the mismatch between \textbf{human input frequency} and \textbf{system processing capacity} creates a phenomenon we term \textit{interaction jitter}:

\begin{itemize}
    \item \textbf{Input Jitter}: Rapid keystrokes in a search box triggering multiple API calls
    \item \textbf{Scroll Jitter}: High-frequency scroll events causing layout thrashing
    \item \textbf{Resize Jitter}: Window resize events firing hundreds of times per second
    \item \textbf{Rendering Jitter}: State updates faster than display refresh rates
\end{itemize}

\subsection{Limitations of Current Approaches}

The de facto solutions—\texttt{debounce} and \texttt{throttle}—suffer from critical limitations:

\begin{enumerate}
    \item \textbf{Ad-hoc Parameter Selection}: Delay values (e.g., 300ms) are chosen by intuition
    \item \textbf{Static Configuration}: Parameters cannot adapt to varying user behavior
    \item \textbf{No Optimality Guarantees}: No theoretical framework guides the choice
    \item \textbf{Hidden Trade-offs}: The latency-stability trade-off is not quantified
\end{enumerate}

\subsection{Our Contributions}

We present the first \textbf{mathematically rigorous framework} for analyzing and optimizing jitter mitigation in web applications:

\begin{enumerate}
    \item \textbf{Formal Model} (Section~\ref{sec:model}): We model user input as a \textit{marked point process} and mitigation strategies as \textit{signal filters}
    
    \item \textbf{Theoretical Analysis} (Section~\ref{sec:theory}): We derive closed-form expressions for optimal parameters under various loss functions
    
    \item \textbf{Adaptive Algorithm} (Section~\ref{sec:algorithm}): We propose an online algorithm that learns optimal parameters from user behavior
    
    \item \textbf{Empirical Validation} (Section~\ref{sec:experiments}): We validate our theory on real-world datasets and demonstrate practical improvements
\end{enumerate}
