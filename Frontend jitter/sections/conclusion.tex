\section{Conclusion and Future Work}

\subsection{Summary}

We presented the first signal-theoretic framework for analyzing jitter mitigation in web applications. Our main contributions are:

\begin{itemize}
    \item \textbf{Formal modeling} of user input as stochastic point processes
    \item \textbf{Closed-form solutions} for optimal debounce and throttle parameters
    \item \textbf{Information-theoretic bounds} on achievable performance
    \item \textbf{Adaptive algorithm} with provable convergence guarantees
    \item \textbf{Empirical validation} demonstrating 23-41\% improvement
\end{itemize}

\subsection{Future Directions}

\begin{enumerate}
    \item \textbf{Deep Learning Integration}: Learn complex user behavior patterns with RNNs/Transformers
    
    \item \textbf{Multi-Modal Signals}: Joint filtering of keyboard, mouse, touch, and voice inputs
    
    \item \textbf{Predictive Filtering}: Use sequence models to anticipate user intent before input completes
    
    \item \textbf{Personalization}: User-specific models that adapt to individual behavior patterns
    
    \item \textbf{Cross-Device Optimization}: Unified framework for desktop, mobile, and embedded systems
    
    \item \textbf{Formal Verification}: Prove correctness properties of filter implementations
\end{enumerate}
